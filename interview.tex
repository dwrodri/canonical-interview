\documentclass{article}
\usepackage[margin=1in]{geometry}
\usepackage{indentfirst}

\title{Written Interview for Python Engineer - Data Center Hardware Integration (Greater Boston Area)}
\author{Name Redacted}
\date{\today}

\begin{document}
\maketitle

\section{Career Development}
\subsection{How would you describe your level of experience managing data servers and network hardware?}

I have an intermediate level of experience managing data servers and network
hardware. I have performed most common tasks related to configuring and
maintaining data servers and network hardware at least once. In the past, I was
a Cisco Certified Entry Network Technician certification and am familiar with
configuring hardware running Cisco IOS.

In both my graduate and undergraduate research labs, I received training in
configuring and installing rack-mounted Linux servers. This training was used
for installing a handful of hardware, mostly GPU servers.  This involved
everything from physical installations to account management to software
installation.

\subsection{How would you describe your level of experience as a software
    engineer?}
% near-senior. I have worked on many different layers of abstraction across
% multiple different projects. There is still much for me to learn, but what I
% don't know I feel capable of teaching myself very quickly.

I believe that my experience has me on track to being a senior engineer in the
next 12-18 months. My prior experiences as a researcher, machine learning
engineer, and consultant have honed my abilities as a software engineer who had
to ship code to production and could be easily maintained by other engineers. I
am comfortable writing code in multiple languages, with my strongest being
Python. Here is an abbreviated list of professional software projects I have worked
on in the past:
\begin{itemize}
    \item Topic classification microservice for a microblog-based social media platform: Python
    \item Optimized post-processing pipeline for real-time object detection in video: Python, C++17, CUDA
    \item A cycle-approximate simulation of a RISC-V CPU down to the basic block level: Golang
\end{itemize}


\subsection{What is your proudest success as an engineer?}

To this day, my produest success is the Discord bot which plays background music
for my friends and while we hang out in voice chat. While it is not my most
technically advanced creation, it is something I made which improves of the
lives of people that are important to me. I often get requests for
feature additions, or bug reports from my friends even when I'm not online,
which tells me that they use it even when I'm not around.

\subsection{Outline the role of an engineering manager in shaping a high
    functioning team.}

An engineering manager shapes a high functioning team in the following ways:
\begin{itemize}
    \item \textbf{Hiring and Firing}: The Engineering Manager is the most aware
          of what skills would be needed to complement the existing team, and thus
          they should have a large role and selecting new hires for their team.
    \item \textbf{Managing Upwards}: The Engineering Manager should have the
          most time to meet with leadership, as implementation work has been delegated
          to the engineers that they manage. This time should be used to ensure that their
          engineers have all the resources (e.g., time, information, infrastructure) they
          need to do well, and that expectations are managed accordingly
    \item \textbf{Mentorship}: The Engineering Manager should serve as a "force
          multiplier" for the engineers they manage, by allocating work in such a way
          that the engineers can grow technically and professionally.
\end{itemize}

\section{Experience} \subsection{Describe your level of experience in Python,
    and how you have attained it.}
% Wrote my first python in 2017, and I have been fond of the language eer since
% I have traslated projects from C++ to Python I have written decrypting
% software that cracked DES ciphers using NumPy I have shipped multiple machine
% learning services written in Python
\subsection{Describe a case where it was very difficult to test code you were
    writing, but you found a reliable way to do it.}

While working as a Machine Learning Engineer, it was common to encounter
code which compiled and ran without software errors, but didn't perform as
expected on certain edge cases. One of the first tasks I set about doing as a
new hire was to develop a special test suite that would allow for us to measure
whether changes to our image processing code affected the results of the object
classification software (In terms of AUC ROC\cite{auc_roc}). This way, every
time a change was made to the image processing code for the classifiers, we could
observe any potentialy degredation in our image detection capabilities and prevent
it from being deployed.

% Real-time Object Detection software tests 10 from training set -- must pass
% 100 unseen examples -- score must always improve
\subsection{When did you start working with Linux? Describe your level of
    experience as a user \& developer on Linux.}
% I started using Linux on a daily basis for university assignments 8 years ago.

I started using Linux on a regular basis 8 years ago for my computer science
coursework and have been using it ever since. I've used a variety of Linux
distros for software development, hosting, to media playback and gaming. I
used a Linux distro to write this document. Ubuntu was the Linux distribution of
choice for the computing department at both my undergraduate and graduate institutions.

\subsection{Describe your experience with networking, storage technologies and
    cloud infrastructure}

I acquired some professional experience with cloud infrastructure by nature of the
small engineering team at my previous employer. This involved the following:
\begin{itemize}
    \item Writing Dockerfiles which assembled a custom SDK for \texttt{x64} and
          \texttt{aarch64} architectures
    \item Managing Artifacts in Google Cloud Platform Artifact Registry
    \item Writing custom CI workflows for GitHub CI and Jenkins
\end{itemize}

\subsection{Describe your experience of large-scale physical server
    installations, including any provisioning, automation and service orchestration
    work.}
% Most of my experience with servers is in the "must fit in a single rack", but
% I have physically installed servers onto racks, made my own CAT5e patch
% cables, and wired them all up to a switch I have worked with Ansible for
% monitoring and logging fleets of production devices
\subsection{Tell us what you believe teams should consider when they build,
    test, run and deliver software.}
% build and release fast and as transparently as possible tests can be binned
% based on necessity: essential, client-driven, and thorough deploy to serve the
% scale you need DEvelopment and Ops are two sides of the same coin
\subsection{How do you think about quality?}

Quality is what the you remember after using a product or service. It's what
provides the sensation of "wow, this really just works" when using a piece of
software. Humans are naturally highly perceptive creatures, and thus high
quality products and services require lots of attention to the finest details.


\subsection{What would you like to achieve in career development and skills
    development?}
% I would like to evolve from a builder into a builder who leads I would like to
% learn from expert leaders to co-develop great open source projects I'd like to
% build strong connections with leaders in the open source community

\section{Education}
\subsection{At high school, how did you fare in mathematics
    and physical sciences? Which were your strongest subjects in the hard sciences,
    and how did you rank in your class?}

I was very strong in math and physical sciences in high school. I competed at the state level
in both domains, and was ranked 18\textsuperscript{th} out of a class of 496 (Top 5\%).

% mathlete + competitive science fair at the regional district Won first place
% at the south carolina regional science fair competition for my analysis of
% fairness in Bitcoin mining pools
\subsection{At high school, what leadership roles did you take on?}

Here is a list of relevant leadership roles from my high school years:
\begin{itemize}
    \item I was the lead editor for all Math and computing content at my highschool's educational newsletter.
    \item I tutored math and Spanish for 4 hours a week as a volunteer.
    \item I represented my high school at the University of South Carolina Summer at Moore business competition.
\end{itemize}
\subsection{What course and university did you choose, and why?}
% Computer Science because I had a passion for understanding how technology
% worked and how to make it more accessible to the world Clemson University
% because I had been awarded the Palmetto Fellows Scholarship

I attended Clemson University and acquired a Bachelors of Science in Computer
Science, because understanding how computers work has been my passion since a
very young age. I was awarded the Palmetto Fellows Scholarship (along with other
awards), and thus was given the opportunity to attend at very low cost, cementing
my choice of university.

\subsection{How did you rank competitively in university? Which were your
    strongest courses, and which did you enjoy the most?}

I was mostly an A/B student in my undergraduate studies, likely in the upper
30\% of the class. Here is a sample of courses where I recieved the highest
grade in my undergraduate:
\begin{itemize}
    \item Network Programming
    \item Design and Analysis of Algorithms
    \item Calculus of Several Variables
\end{itemize}

I really enjoyed courses that allowed me to take skills I had developed earlier on and build projects, here are some examples:
\begin{itemize}
    \item 2-D Game Engine Construction
    \item Applied Data Science
    \item Robotics
\end{itemize}

My academic showing during my Master of Science in Computer Engineering was much stronger, with a GPA of 3.8/4.0.

% I was an A/B student Strongest courses were very freeform: research, senior
% seminars, robotics
\subsection{At high school and university, describe your achievements that were
    considered exceptional by colleagues and staff.}

\begin{itemize}
    \item \textbf{Clemson School of Computing Best Research}  I built monitoring infrastructure for
          memory accesses in Nvidia's Unified Virtual Memory system, and eventually
          presented the research at Super Computing 2019.
    \item \textbf{1\textsuperscript{st} Place South Carolina Junior Academy of Sciences, Math
              and Computing} I studied reward allocation behavior in Bitcoin mining pools
          from 2013-2014 to determine if certain policies were more beneficial to
          small, individual mining participants such as myself.
    \item \textbf{LQTLD3} I wrote a map-parsing library for a robotics project
          that allowed for the path-finding algorithm to determine available adjacent
          space in constant time, and then made it open source. My implementation is based
          a paper by Kunio Aizawa and Shojiro Tanaka\cite{4538229} and the repository is hosted
          on GitHub\cite{lqtld3}.

\end{itemize}
\section{Context}
\subsection{How are you involved in open source software? Describe any
    significant contributions to open source (with links where possible)}

As a professional developer, I am a contributor to Akita\cite{akita}, which is
an MIT-licensed CPU/GPU simulation framework for building and testing
microarchitectures against different workloads. I became a contributor for Akita
through my graduate research, where I used Akita to build a RISC-V emulator.

As a user, I see it as my duty to file issues and work on improving the product
when possible, however previous employers chose not allocate time towards
merging our code upstream in cases where we did work with open source software.

\subsection{What do you think are the key ingredients of a successful open
    source project?}

\begin{itemize}
    \item \textbf{Effective leadership}, who is capable of communicating
          priorities based on a strong vision, and running the project well at it's
          current stage.
    \item \textbf{Passionate builders}, who are technically capable to convert priorities
          into features. Often in the early stages, the leadership is also building.
    \item \textbf{Transparency}. Open source software wins because it is
          extensible and transparent. It is easiest to build on software when the
          person building on top of the software has all of the control they need to
          get the job done. There was a great essay on this recently by the Co-founder
          of Lago\cite{oss_cheap}.
\end{itemize}

\subsection{Why do you most want to work for Canonical?}

% Mission driven Work across various layers of the stack Build a network with
% the open source community

I firmly believe in Canonical's mission to make the frontier of technology an accessible
resource to all. I originally applied to grad school so that I could help make advanced
machine learning methods more accessible. Building infrastructure with Canonical is a
solid step towards making my mission a reality.

\subsection{Which other companies are building the sort of products you would
    like to work on?}
Here's what I seek out in terms of projects:
\begin{itemize}
    \item Technical Challenge to improve / maintain
    \item Provides utility to people
    \item Generally related to a subject I'm experienced with (e.g., Hardware/Software, Audio/Video, Machine Learning)
\end{itemize}
Here are some examples of companies that make products which meet those
requirements:
\begin{itemize}
    \item Apple (WebKit, Clang/LLVM)
    \item Microsoft (GitHub, Visual Studio Code)
    \item HuggingFace (HuggingFace Hub)
\end{itemize} \subsection{What do you think Canonical needs to improve in its
    engineering and products?}

I think Canonical would benefit from

\subsection{Who do you think are key competitors to Canonical? How do you think
    Canonical should plan to win that race?}

The key competitors to Canonical are RedHat/IBM, SUSE, and to a lesser extent,
HashiCorp. These companies compete with Canonical by providing software which
attempts to simplify and streamline managing enterprise software +
infrastructure. I think beating out this portion of the competition comes down
to providing a superior UX on the handful of workflows that matter the most to a
target userbase.


\bibliographystyle{plain}
\bibliography{refs}
\end{document}