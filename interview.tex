\documentclass{article}
\usepackage[margin=1in]{geometry}
\usepackage{multicol}
\usepackage{indentfirst}

\title{Written Interview for Python Engineer - Data Center Hardware Integration (Greater Boston Area)}
\author{Name Redacted}
\date{\today}

\begin{document}
\maketitle

\section{Career Development}
\subsection{How would you describe your level of experience managing data servers and network hardware?}

I have an intermediate level of experience managing data servers and network
hardware. I have performed most common tasks related[a] to configuring and
maintaining data servers and network hardware at least once. In the past, I
earned a Cisco Certified Entry Network Technician certification and am familiar
with configuring hardware running Cisco IOS.

In both my graduate and undergraduate research labs, I received training in
configuring and installing rack-mounted Linux servers. The training was very
comprehensive, covering physical installations, account management, and software
installation. I applied this skillset to install a handful of hardware, mostly
GPU servers.

\subsection{How would you describe your level of experience as a software
  engineer?}

I currently have 3 years of experience building software as a machine learning
engineer, researcher, and conultant. My prior experiences  have honed my ability
to ship maintainable code to production. I am comfortable writing code in
multiple languages, with my strongest being Python. Here is an abbreviated list
of professional software projects I have worked on in the past:

\begin{itemize}
  \item Topic classification microservice for a microblog-based social media platform: Python
  \item Optimized post-processing pipeline for real-time object detection in video: Python, C++17, CUDA
  \item A cycle-approximate simulation of a RISC-V CPU down to the basic block level: Golang
\end{itemize}

\subsection{What is your proudest success as an engineer?}

To this day, my proudest success is the Discord bot which plays background music
for my friends and while we hang out in voice chat. While it is not my most
technically advanced creation, it is something I made which improves of the
lives of people that are important to me. I often get requests for feature
additions and bug reports from my friends even when I'm not online, which tells
me that they use it even when I'm not around.

\subsection{Outline the role of an engineering manager in shaping a high
  functioning team.}

An engineering manager shapes a high functioning team in the following ways:
\begin{itemize}
  \item \textbf{Hiring and Firing}: The Engineering Manager is the most aware of
        what skills would be needed to complement the existing team, and
        thus they should have a key role in selecting new hires for their
        team.

  \item \textbf{Managing Upwards}:The Engineering Manager should have the time
        to meet with leadership. This time should be used to
        ensure that their engineers have all the resources (e.g., time,
        information, infrastructure) they need to do well, and that
        expectations are managed accordingly.

  \item \textbf{Mentorship}: The Engineering Manager should serve as a "force
        multiplier" for the engineers they manage, by allocating work in such a way
        that the engineers can grow technically and professionally.

\end{itemize}

\section{Experience} \subsection{Describe your level of experience in Python,
  and how you have attained it.}

The first time I used Python to create data visualizations for a research paper about 7 years
ago, and it has been my favorite language ever since. Today, I'd
I consider myself very experienced in Python. Here is a sample of work
I've done:

\begin{itemize}
  \item A Proof-of-Concept Flush+Reload attack on AES encryption, based on work
        by Yaroum and Falkner~\cite{flushreload}.
  \item A monitoring service written in Python that uses FFmpeg to check the
        integrity of 30-second videos being uploaded to S3-compatible cloud service.
  \item A text classification service for short-form blog posts, that is
        currently deployed in the production environment for a small social media app
\end{itemize}

I've also written blog posts where I've dissected Python bugs at the bytecode
level, and maintained Python codebases that I haven't built/written
myself.

\subsection{Describe a case where it was very difficult to test code you were
  writing, but you found a reliable way to do it.}

While working as a Machine Learning Engineer, it was common to encounter
code which compiled and ran without software errors, but didn't perform as
expected on certain edge cases. One of the first tasks I set about doing as a
new hire was to develop a special test suite that would allow for us to measure
whether changes to our image processing code affected the results of the object
classification software (In terms of AUC ROC\cite{auc_roc}). This way, every
time a change was made to the image processing code for the classifiers, we could
observe any potentialy degredation in our image detection capabilities and prevent
it from being deployed.

% Real-time Object Detection software tests 10 from training set -- must pass
% 100 unseen examples -- score must always improve
\subsection{When did you start working with Linux? Describe your level of
  experience as a user \& developer on Linux.}

I started using Linux on a regular basis 8 years ago for my computer
science coursework and have been using it ever since. I've used a
variety of Linux distros for software development, hosting, media playback, gaming, and more.
Ubuntu was the Linux distribution of choice for the computing department
at both my undergraduate and graduate institutions.

\subsection{Describe your experience with networking, storage technologies and
  cloud infrastructure}

I acquired professional experience with cloud infrastructure by working on a
small engineering team at my previous employer. My role as a machine learning engineer
involved the following:
\begin{itemize}
  \item Writing Dockerfiles which assembled a custom SDK for \texttt{x64} and
        \texttt{aarch64} architectures
  \item Managing Artifacts in Google Cloud Platform Artifact Registry
  \item Writing custom CI workflows for GitHub CI and Jenkins
\end{itemize}

\subsection{Describe your experience of large-scale physical server
  installations, including any provisioning, automation and service orchestration
  work.}

My experience dealing with physical server installations is at the scale of
adding, removing, and troubleshooting connections on one or two racks of
servers. Notably, I collaborated with a professor to install several Lambda Labs
Hyperplane 4-A100 servers into the rack at our university lab, connect them to
the network, and create all of the user profiles.

When it comes to service orchestration, I configured Portainer to deploy and
monitor our computer vision software on an array of several hundred x64 small
form factor PCs. Portainer was connected to our cloud infrastructure, where we
build container images for our software that Portainer could deploy to our
hardware. We also configured log aggregation, downtime notifications, and
enabled support for remote login when specific machines were malfunctioning.

\subsection{Tell us what you believe teams should consider when they build,
  test, run and deliver software.}
% build and release fast and as transparently as possible tests can be binned
% based on necessity: essential, client-driven, and thorough deploy to serve the
% scale you need DEvelopment and Ops are two sides of the same coin
\begin{itemize}
  \item Above all, adapt the process to the nature of the problem. The development cycle for a signal processing library on a 3-person team is different from infrastructure software at a 10,000-person company.
  \item Collect and evaluate requirements and their respective priorities early and often. They are the most
        useful guide for determining what needs to be built before delivery. They also determine how you build
        your test suite.
  \item Your first delivery should be when the subset requirements deemed critical are built and tested.
  \item A great way to stay on schedule and stay with the mantra of ”ship early and ship often” is to always
        build according to the scale of your users.
  \item You shouldn’t build a feature without a test plan written down somewhere.
  \item Every developer working on the product should have the means to build, test, and run their whole
        product without involving another human in the process.
\end{itemize}

Here are some texts that I cite as the basis for my beliefs:

\begin{itemize}
  \item The Mythical Man-Month by Fred Brooks~\cite{mmm}
  \item The Joel Test: 12 Steps to Better Code by Joel Spolsky\cite{spolsky_code}
  \item Code Craft: The Practice of Writing Excellent Code by Pete Goodliffe\cite{codecraft}
  \item Game Engine Architecture by Jason Gregory, specifically the Chapter 3 of Part I \cite{key}
\end{itemize}

\subsection{How do you think about quality?}

Quality is what you remember after using a product or service. It's what
provides the sensation of "wow, this really just works" when using a piece of
software. Humans are highly perceptive creatures, and thus high quality products
and services require lots of attention to the finest details.


High quality software development follows a very similar pattern. The codebase
is managed with a version control system. There should also be systems for
automated building, testing, and delivery in place. If a new person joins the
software development process, their ability to effectively contribute should be
proportional to their related experience. The documentation for the software
should comprehensively cover the problem(s) that the software intends to solve,
and all of the details related to how the software solves them.


My experience with high quality software development comes from my time working
as a consultant building natural language processing software. I was building a
service that was incorporated into a codebase that was already many hundreds of
thousands of lines, but was easy to digest when broken into small parts. It was
easy to delineate what information I needed to know to solve the problem at
hand, thanks to the thorough documentation and great communication skills of the
engineers I was assisting as a consultant. Once my code was written and the
service was deployed, it ran very efficiently on the hardware provisioned.


\subsection{What would you like to achieve in career development and skills
  development?}

\begin{itemize}
  \item Build a strong network in the open source community
  \item Deliver excellent software that makes advanced technology more
        accessible to all
  \item Build a platform as a talented engineer who can both build great
        products and communicate effectively for a global audience.
\end{itemize}

\section{Education}
\subsection{At high school, how did you fare in mathematics
  and physical sciences? Which were your strongest subjects in the hard sciences,
  and how did you rank in your class?}

I was very strong in math and physical sciences in high school. I competed at the state level
in both domains, and was ranked 18\textsuperscript{th} out of a class of 496 (Top 5\%). Here is
a sample of courses where I attained the highest mark:
\begin{multicols}{2}
  \begin{itemize}
    \item Advanced Placement Calculus BC
    \item Advanced Placement Computer Science
    \item Physics
    \item Chemistry
    \item Biology
    \item Algebra
    \item Geometry
  \end{itemize}
\end{multicols}

% mathlete + competitive science fair at the regional district Won first place
% at the south carolina regional science fair competition for my analysis of
% fairness in Bitcoin mining pools
\subsection{At high school, what leadership roles did you take on?}

Here is a list of relevant leadership roles from my high school years:
\begin{itemize}
  \item I was the lead editor for all Math and computing content at my highschool's educational newsletter.
  \item I tutored math and Spanish for 4 hours a week as a volunteer.
  \item I represented my high school at the University of South Carolina Summer at Moore business competition.
\end{itemize}

\subsection{What course and university did you choose, and why?}

I attended Clemson University and acquired a Bachelors of Science in Computer
Science, because understanding how computers work has been my passion since a
very young age. I was awarded the Palmetto Fellows Scholarship (along with other
awards), and thus was given the opportunity to attend at very low cost, cementing
my choice of university.

\subsection{How did you rank competitively in university? Which were your
  strongest courses, and which did you enjoy the most?}

I was an A/B student in my undergraduate studies, with a cumulative GPA of
3.44/4.0. Here is a sample of courses where I recieved the highest grade in my
undergraduate:

\begin{itemize}
  \item Network Programming
  \item Design and Analysis of Algorithms
  \item Calculus of Several Variables
\end{itemize}

I really enjoyed courses that allowed me to take skills I had developed earlier
on and build projects, here are some examples:

\begin{itemize}
  \item 2-D Game Engine Construction
  \item Applied Data Science
  \item Robotics
\end{itemize}

My academic showing during my Master of Science in Computer Engineering was much stronger, with a GPA of 3.8/4.0.

% I was an A/B student Strongest courses were very freeform: research, senior
% seminars, robotics
\subsection{At high school and university, describe your achievements that were
  considered exceptional by colleagues and staff.}

\begin{itemize}
  \item \textbf{Clemson School of Computing Best Research} I built monitoring
        infrastructure to trace memory accesses in Nvidia's Unified Virtual
        Memory system, and eventually presented the research at Super
        Computing 2019.

  \item \textbf{1\textsuperscript{st} Place South Carolina Junior Academy of Sciences, Math
          and Computing} I studied reward allocation behavior in Bitcoin mining pools
        from to determine if certain policies were more beneficial to
        small, individual mining participants such as myself.

  \item \textbf{LQTLD3} I wrote a map-parsing library for a robotics project
        that allowed for the path-finding algorithm to determine available adjacent
        space in constant time, and then made it open source. My implementation is
        based on a paper by Kunio Aizawa and Shojiro\cite{4538229} and the repository
        is hosted on GitHub.

\end{itemize}
\section{Context}
\subsection{How are you involved in open source software? Describe any
  significant contributions to open source (with links where possible)}

As a professional developer, I am a contributor to Akita\cite{akita}, which is
an MIT-licensed CPU/GPU simulation framework for building and testing
microarchitectures against different workloads. I became a contributor for Akita
through my graduate research, where I used Akita to build a RISC-V emulator.

As a user, I see it as my duty to file issues and work on improving the product
when possible, however previous employers chose not allocate time towards
merging our code upstream in cases where we did work with open source software.

\subsection{What do you think are the key ingredients of a successful open
  source project?}

\begin{itemize}
  \item \textbf{Effective leadership}, who is capable of communicating
        priorities based on a strong vision, and running the project well at it's
        current stage. Effective leadership for open source manifests itself in the
        form of open governance and a diverse base of contributors.


  \item \textbf{Passionate builders}, who are technically capable to convert
        priorities into features. Often in the early stages, the leadership is also
        building. Passionate builders are going to be capable and willing to put in
        the extra time to create a high quality project.

  \item \textbf{Transparency}. Open source software wins because it is
        extensible and transparent. It is easiest to build on software when the
        person building on top of the software has all of the control they need to
        get the job done. There was a great essay on this recently by the Co-founder
        of Lago\cite{oss_cheap}.
\end{itemize}

\subsection{Why do you most want to work for Canonical?}


I firmly believe in Canonical's mission to make cloud infrastructure open and
accessible. I originally applied to graduate school so that I could help make
advanced machine learning methods more accessible. Building infrastructure with
Canonical is a solid step towards making my mission a reality.

\subsection{Which other companies are building the sort of products you would
  like to work on?}
Here's what I seek out in terms of projects:
\begin{itemize}
  \item Technical Challenge to build / improve
  \item Provides utility to people
  \item Generally related to a subject I'm experienced with (e.g., Hardware/Software, Audio/Video, Machine Learning)
\end{itemize}
Here are some examples of companies that make products which meet those
requirements:
\begin{itemize}
  \item Apple (WebKit, Clang/LLVM)
  \item Microsoft (GitHub, Visual Studio Code)
  \item HuggingFace (HuggingFace Hub)
\end{itemize} \subsection{What do you think Canonical needs to improve in its
  engineering and products?}

I think Canonical would benefit from a greater emphasis on "turnkey"
workflows for certain classes of projects. I think an important test
would be to create a non-trivial web service like the video streaming
service described in this blog post\cite{streaming_blog} using Canonical's
own managed service, and then see what it's like to build the same
service using the offerings of the competition.

My opinion comes from positive experiences I've had with
software services in the past: Namely, Heroku's famous
\texttt{git push heroku master} workflow, as well as the workflow for setting
up static websites on GitHub Pages. These workflows stick out in my mind
as being extremely accessible, much like my first experience installing
Ubuntu long ago.

Both of these examples cater to small developer shops and hobbyists who are
price sensitive. If these markets aren't appealing to Canonical , then
alternatively, Canonical could place a focus on consulting for the
cybersecurity/privacy (SOC2 compliance, simplified IAM, HIPAA compliance)
domain.

\subsection{Who do you think are key competitors to Canonical? How do you think
  Canonical should plan to win that race?}

The key competitors to Canonical are RedHat/IBM, SUSE, and HashiCorp. These
companies compete with Canonical by providing software which attempts to
simplify and streamline managing enterprise software + infrastructure. I think
beating out this portion of the competition comes down to providing a superior
UX on the handful of workflows that matter the most to a target userbase.


\bibliographystyle{plain}
\bibliography{refs}
\end{document}