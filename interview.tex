\documentclass{article}
\usepackage[margin=0.5in]{geometry}

\title{Written Interview for Python Engineer - Data Center Hardware Integration (Greater Boston Area)}
\author{Derek Rodriguez (derek@derekrodriguez.dev)}
\date{\today}

\begin{document}
\maketitle

\section{Career Development}
\subsection{How would you describe your level of experience managing data servers and network hardware?}
% intermediate, have done most tasks at least once, but would still benefit from the guidance of a manager or senior lead on large projects
\subsection{How would you describe your level of experience as a software engineer?}
% near-senior. I have worked on many different layers of abstraction across multiple different projects. There is still much for me to learn, but what I don't know I feel capable of teaching myself very quickly.
\subsection{What is your proudest success as an engineer?}
% Either the Discord bot or when I found corruption bugs in prod
\subsection{Outline the role of an engineering manager in shaping a high functioning team.}
% Goes to bat for engineers who report to them to higher ups for needs
% Setting priorities based on OKRs
% Can step in as a fellw engineer to assist when necessary: setting requirement scope
% Helps with hirng and firing

\section{Experience}
\subsection{Describe your level of experience in Python, and how you have attained it.}
% Wrote my first python in 2017, and I have been fond of the language eer since
% I have traslated projects from C++ to Python
% I have written decrypting software that cracked DES ciphers using NumPy
% I have shipped multiple machine learning services written in Python
\subsection{Describe a case where it was very difficult to test code you were writing, but you found a reliable way to do it.}
% Real-time Object Detection software tests
% 10 from training set -- must pass
% 100 unseen examples -- score must always improve
\subsection{When did you start working with Linux? Describe your level of experience as a user \& developer on Linux.}
% I started using Linux on a daily basis for university assignments 8 years ago.
\subsection{Describe your experience with networking, storage technologies and cloud infrastructure}
% I once held a valid CCENT certification (although now it expired)
% I acquired some professional experience with cloud infrastructure by nature of the small size of our engineering department
\subsection{Describe your experience of large-scale physical server installations, including any provisioning, automation and service orchestration work.}
% Most of my experience with servers is in the "must fit in a single rack", but I have physically installed servers onto racks, made my own CAT5e patch cables, and wired them all up to a switch
% I have worked with Ansible for monitoring and logging fleets of production devices
\subsection{Tell us what you believe teams should consider when they build, test, run and deliver software.}
% build and release fast and as transparently as possible
% tests can be binned based on necessity: essential, client-driven, and thorough
% deploy to serve the scale you need
% DEvelopment and Ops are two sides of the same coin
\subsection{How do you think about quality?}
% Quality is always relative: everything is a comparison, so it's important that 
\subsection{What would you like to achieve in career development and skills development?}
% I would like to evolve from a builder into a builder who leads
% I would like to learn from expert leaders to co-develop great open source projects
% I'd like to build strong connections with leaders in the open source community

\section{Education}
\subsection{At high school, how did you fare in mathematics and physical sciences? Which were your strongest subjects in the hard sciences, and how did you rank in your class?}
% mathlete + competitive science fair at the regional district 
% Won first place at the south carolina regional science fair competition for my analysis of fairness in Bitcoin mining pools
\subsection{At high school, what leadership roles did you take on?}
% co-editor of the STEM magnet newspaper
\subsection{What course and university did you choose, and why?}
% Computer Science because I had a passion for understanding how technology worked and how to make it more accessible to the world
% Clemson University because I had been awarded the Palmetto Fellows Scholarship
\subsection{How did you rank competitively in university? Which were your strongest courses, and which did you enjoy the most?}
% I was an A/B student
% Strongest courses were very freeform: research, senior seminars, robotics
\subsection{At high school and university, describe your achievements that were considered exceptional by colleagues and staff.}
% Best research in the CS department
% 

\section{Context}
\subsection{How are you involved in open source software?}
% I have begun immersing myself 
\subsection{Describe any significant contributions to open source (with links where possible)}
% Akita
\subsection{What do you think are the key ingredients of a successful open source project?}
% Good stewardship: Guido van Rossum, Linus
\subsection{Why do you most want to work for Canonical?}
% Mission driven
% Work across various layers of the stack
% Build a network with the open source community
\subsection{Which other companies are building the sort of products you would like to work on?}
Here's what I seek out in terms of projects:
\begin{itemize}
    \item Technical Challenge to improve / maintain
    \item Provides utility to people
    \item Generally, makes the world a better place
\end{itemize}
Here are some examples of companies that make products which meet those requirements:
\begin{itemize}
    \item Apple (iOS, macOS, XCode, iMessage)
    \item Microsoft (Azure, GitHub, Visual Studio Code)
    \item HuggingFace (HuggingFace Hub)
\end{itemize}
\subsection{What do you think Canonical needs to improve in its engineering and products?}
% Canonical needs to improve the pipeline for getting hobbyist devs and solopreneurs onto its enterprise offerings
\subsection{Who do you think are key competitors to Canonical? How do you think Canonical should plan to win that race?}
% RedHat/IBM and Hetzner 
\end{document}